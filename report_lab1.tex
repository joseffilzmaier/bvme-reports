%Schriftgr��e, Layout, Papierformat, Art des Dokumentes
\documentclass[12pt,oneside,a4paper]{scrartcl}

%Einstellungen der Seitenr�nder
\usepackage[left=3cm,right=2cm,top=2cm,bottom=2cm,includeheadfoot]{geometry}

\usepackage{lastpage}

%neue Rechtschreibung
\usepackage{ngerman}

%Umlaute erm�glichen
\usepackage[latin1]{inputenc}

%Kopf- und Fu�zeile
\usepackage{fancyhdr}
\pagestyle{fancy}
\fancyhf{}

%Kopfzeile links bzw. innen
\fancyhead[L]{BMV Labor 1}
%Kopfzeile mittig
\fancyhead[C]{Filzmaier Josef, Michael Sieberer}
%Kopfzeile rechts bzw. au�en
\fancyhead[R]{\today}
%Linie oben
\renewcommand{\headrulewidth}{0.5pt}

%Fu�zeile rechts bzw. au�en
\fancyfoot[R]{\thepage/\pageref{LastPage}}
%Fu�zeile links
\fancyfoot[L]{Matr.Nr.:1030462, 1531366}
%Linie unten
\renewcommand{\footrulewidth}{0.5pt}

\begin{document}

\section{Exercise 1 - Image stitching}

Two images from two different perspectives containing the same chess board are taken with two different cameras (Focus length $12.5mm$ and $6.5mm$).
Four key locations within the chess board are chosen in the first image which are then mapped to points within the second image.
Several scenarios are evaluated:
\begin{itemize}
 \item All four points are corresponding within the chess board over a large area
 \item All four points are corresponding within the chess board over a small area
 \item One of the points is chosen differently within image one than in image two
 \item All points are on one line except for one
\end{itemize}

\section{Exercise 2 - Auto Focus}

In this exercise we had to take a sequence of $N=10$ images with linearly varying focus and the determine which image has the best focus option.
In order to solve this task we implemented the following idea:

\begin{itemize}
 \item Convert the colored image to the corresponding grey value image
 \item Calculate the histogram of the grey value image
 \item Sort the histogram values decreasingly
 \item Sum up the first $1000$ elements of the sorted histogram and normalize
\end{itemize}

\section{Exercise 3 - Sensor Dynamics}

In this exercise a black and white colored disk with an embedded EMT logo is spun up using an electronic motor.
The goal is to take a photo of the EMT logo while the disk is spinning.
This is done using a LED flashlight which is triggered by a rotation sensor and synchronously flashes the disk.
Therefore the logo is perceived static because the light highlights the disk always when the logo is at the same place.

\subsection{Global Shutter}

A global shutter camera measures light intensity with all pixel sensors at the same point in time.

\subsection{Rolling Shutter}

A rolling shutter camera measures light intensity line by line.
This can cause certain artifacts with fast moving objects.

\section{Exercise 4 - Perspective Invariants}

In this exercise we take a photo of an prepared image with four points which have a certain distance to each other.
The points are prepared to easily be detected by a \texttt{cornerMark} function which is given in the framework.
Two different cameras with focal lenth of $4.3mm$ and $12.5mm$ were used.\\

\noindent
The claim to verify is that formula \ref{eq:cross_ratio} is independent of the perspective the picture is taken from as well as independent of the used focal length.

\begin{equation}
 CR_{A,B,C,D} = \frac{\overline{AC}}{\overline{BC}} : \frac{\overline{AD}}{\overline{BD}}
 \label{eq:cross_ratio}
\end{equation}

\subsection{Measuring by hand}

\begin{minipage}{0.48\textwidth}
  $$\overline{AB} = 42mm$$
  $$\overline{BC} = 82mm$$
  $$\overline{CD} = 123mm$$
\end{minipage}
\begin{minipage}{0.48\textwidth}
  $$\overline{AC} = 123mm$$
  $$\overline{BD} = 205mm$$
  $$\overline{AD} = 246mm$$
\end{minipage}

\hspace{0.5mm}

$$CR_{A,B,C,D} = \frac{\overline{AC}}{\overline{BC}} : \frac{\overline{AD}}{\overline{BD}} = \frac{123mm}{82mm} : \frac{246mm}{205mm} = 1.25$$

\subsection{Measuring using the cornerMark Matlab algorithm}



\end{document}
